\documentclass[a4paper,12pt]{article} % тип документа

% report, book

%  Русский язык

\usepackage[T2A]{fontenc}			% кодировка
\usepackage[utf8]{inputenc}			% кодировка исходного текста
\usepackage[english,russian]{babel}	% локализация и переносы

\usepackage[margin=0.5in]{geometry}
\usepackage{xcolor}
\usepackage{hyperref}
 
 % Цвета для гиперссылок
\definecolor{linkcolor}{HTML}{799B03} % цвет ссылок
\definecolor{urlcolor}{HTML}{799B03} % цвет гиперссылок
 
\hypersetup{pdfstartview=FitH,  linkcolor=linkcolor,urlcolor=urlcolor, colorlinks=true}

% Математика
\usepackage{amsmath,amsfonts,amssymb,amsthm,mathtools} 


\usepackage{wasysym}
% мои добьавки-----------------------------------------
\usepackage{ dsfont }
\usepackage{listings}

%мое----------------------------------------------------
%Заговолок
\author{Попов Николай}
\title{Домашнее задание 1\LaTeX{}}

%--------------------
\usepackage{tikz}

\begin{document}

\section*{Задача 1}
Посчитаем произведение матриц 
\[
A = \dfrac{1}{n}M_n(w)M_n(w^{-1})=\left[\sum_{k=0}^{n-1}{w_n^{ik}w_n^{-kj}}\right]_{i,j=0\cdots n-1} =
\left[\sum_{k=0}^{n-1}{\left(w_n^{i-j}\right)^{k}}\right]_{i,j=0\cdots n-1} 
\]
Т.~к. $i$ и $j$ меньше $n$, то если $i = j$, то сумма равна $n$ (элемент на диагонали матрицы), иначе  
она равна нулю по лемме 3 о сложении из файла Горбунова~Э. Тогда $A = I_n$ и формула для обратной матрицы верна. Теми же рассуждениями о сумме получаем

\[
[M_n^2(w)]_{ij}=\sum_{k=0}^{n-1}{w^{ik}w^{kj}}=\sum_{k=0}^{n-1}{\left(w^{i+j}\right)^k} = \begin{vmatrix}
0,~if~(i+j)\neq 0~mod~n\\
n,~if~(i+j)=0~mod~n\\
\end{vmatrix} 
\]

\[
M_n^2(w)=
\begin{vmatrix}
n&0&0&0&  \cdots&0&0&0&0\\
0&0&0&0& \cdots&0&0&0&n\\
0&0&0&0& \cdots&0&0&n&0\\
0&0&0&0& \cdots&0&n&0&0\\
0&0&0&0& \cdots&n&0&0&0\\
\vdots&\vdots&\vdots&\vdots& \cdots&\vdots&\vdots&\vdots&\vdots\\
0&0&0&n& \cdots&0&0&0&0\\
0&0&n&0& \cdots&0&0&0&0\\
0&n&0&0& \cdots&0&0&0&0\\
\end{vmatrix}
\]

\[
M_n^4(w)=\left(M_n^2(w)\right)^2=n^2I_n
\]



\section*{Задача 2}
${y}_A$ обозначим вектор значений многочлена $A(x)$ и ${y}_B$ аналогично. \\
$a_A = (2~3~0~1~0~0~0~0)$, $a_B = (2~0~3~3~0~0~0~0)$  ~---~ векторы их коэффициентов. При рекурсивном спуске вычисления произведения матрицы Вандермонда на вектор считаем размер метрицы 2 константой.\\
$a_1=(2~0), a_2=(0~0), a_3= (3~0), a_4 =(1~0),$ \\
$ b_1=(2~0), b_2=(3~0), b_3= (0~0), a_4 =(3~0$\\

\noindent $M_2a_1=(2~2)^T$, $M_2a_2=(0~0)^T \Rightarrow$
$M_4(2~0~0~0)^T=(2~2~2~2)^T$

\noindent $M_2a_3=(3~3)^T$, $M_2a_4=(2~2)^T \Rightarrow$
$M_4(3~1~0~0)^T=(3~3~3~3)^T + (1~i~-1~-i)^T= $ $=(4~3+i~2~3-i)^T$

\noindent \[y_A=M_8(2~3~0~1~0~0~0~0)^T=
(2~2~2~2~2~2~2~2)^T+\]\[+(4;~\frac{1}{\sqrt{2}}(3+i)(1+i);~2i;~\frac{1}{\sqrt{2}}(3-i)(i-1);~-4;~-\frac{1}{\sqrt{2}}(3+i)(1+i);~-2i;~-\frac{1}{\sqrt{2}}(3-i)(i-1))=\]
\[=
(6;~2+\sqrt{2}+2\sqrt{2}i;~2+2i;~2-\sqrt{2}+2\sqrt{2}i;~-2;~2-\sqrt{2}-2\sqrt{2}i;~2-2i;~2+\sqrt{2}-2\sqrt{2}i)\]\\
По тем же правилам получаем для второго многочлена 
\[
y_B = M_8(2~0~3~3~0~0~0~0)^T=
\]
\[
(8;~
2-3/\sqrt{2}+3i+3i/\sqrt{2};~
-1-3i;~
2+3/\sqrt{2}-3i+3i/\sqrt{2};~
2;~
2+3/\sqrt{2}+3i-3i/\sqrt{2};~
-1+3i;~\]\[
2-3/\sqrt{2}-3i-3i/\sqrt{2})
\]

Пусть $A(x)B(x)=C(x)$. Почленно перемножив получаем $y_C$\\
\[
(48;~
-9-4\sqrt{2}+ 7\sqrt{2}i-3i;~
4-8i;~
-9+4\sqrt{2}+7\sqrt{2}i+3i;~
-4;~
-5 + 7\sqrt{2}+3i-10\sqrt{2}i;~
4+8i;~
-5 -7\sqrt{2}-3i-10\sqrt{2}i)
\]

Далее аналогично перемножаем $a_C=\dfrac{1}{8}M_8(w^{-1})y_C = (4; 6; 6; 17; 9; 3; 3; 0)^T$

\section*{Задача 3}
Решим данную задачу, разделяя ее на каждом шаге рекурсивного спуска на две подзадачи равного размера (для этого предположим, что $n$ есть степень двойки, а затем, используя монотонность числа операций $T(n)$ и приблизив $n$ степенями двойки получим тот же результат для произвольного $n$). На каждом шаге рекурсивного подьема вычисляем произведение двух многочленов :
\[
T(n)=2T\left(\dfrac{n}{2}\right)+\Theta\left(\dfrac{n}{2}\log{\dfrac{n}{2}}\right)= \dfrac{n}{2}\sum_{k=0}^{\log_2{n}}{(\log{n}-k\log{2})}=\dfrac{1}{2}n\log^2{n}-\dfrac{\log{2}}{2}n	\sum_{k=0}^{\log_2{n}}{k}=\]
\[=\dfrac{1}{2}n\log^2{n}-\dfrac{\log{2}}{2}n\log{n}(\log{n}+1)=\Theta(n\log^2{n})
\]


\section*{Задача 4} 
 $F-~$матрица Фурье, $\Lambda$ есть диагональная матрица из собственных векторов циркулянтной матрицы $C$. По формуле для собственных значений матрицы $C$:\\ $\lambda_i = c_0 + c_1(w_n^i)+c_2(w_n^i)^2+\cdots+c_{n-1}(w_n^i)^{n-1}$ и $w=i$ получаем $\lambda_0 = 15, \lambda_1 = -3-6i, \lambda_2 = -5, \lambda_3 = -3+6i$\\
$
Cx=b \Rightarrow x = C^{-1}x
$ .С семинара $FC=\Lambda F \Rightarrow C=F^{-1}\Lambda F \Rightarrow C^{-1}=F^{-1}\Lambda^{-1}F \Rightarrow x=F^{-1}\Lambda^{-1}Fb$, что вычислим в порядке $x=F^{-1}(\Lambda^{-1}(Fb))$ по ДПФ:

\[
x = \dfrac{1}{4}
\begin{pmatrix}
1&1&1&1\\
1&-i&-1&i\\
1&-1&1&-1\\
1&i&-1&-i\\
\end{pmatrix}
\begin{pmatrix}
\dfrac{1}{15}&0&0&0\\
0&\dfrac{-1}{3+6i}&0&0\\
0&0&\dfrac{-1}{5}&0\\
0&0&0&\dfrac{1}{6i-3}\\
\end{pmatrix}
\begin{pmatrix}
1&1&1&1\\
1&i&-1&-i\\
1&-1&1&-1\\
1&-i&-1&i\\
\end{pmatrix}
\begin{pmatrix}
16\\
8\\
4\\
2\\
\end{pmatrix}=
\]
\[
= \dfrac{1}{4}
\begin{pmatrix}
1&1&1&1\\
1&-i&-1&i\\
1&-1&1&-1\\
1&i&-1&-i\\
\end{pmatrix}
\begin{pmatrix}
\dfrac{1}{15}&0&0&0\\
0&\dfrac{-1}{3+6i}&0&0\\
0&0&\dfrac{-1}{5}&0\\
0&0&0&\dfrac{1}{6i-3}\\
\end{pmatrix}
\begin{pmatrix}
30\\
12+6i\\
10\\
12-6i\\
\end{pmatrix}=
\]
\[
=\dfrac{1}{4}
\begin{pmatrix}
1&1&1&1\\
1&-i&-1&i\\
1&-1&1&-1\\
1&i&-1&-i\\
\end{pmatrix}
\begin{pmatrix}
2\\
(-8+6i)/5\\
-2\\
(-8-6i)5\\
\end{pmatrix}=\dfrac{1}{5}
\begin{pmatrix}
-4\\
8\\
4\\
2\\
\end{pmatrix}
\]

\section*{Задача 5} 
1)$
[FFT(x)]_{i}=[Fx]_i=\sum_{k=0}^{n-1}{x_k(w_n^i)^k} = \lambda_i
$~---~ координата вектора (равна собственному значению матрицы $circ(x)$).\\

\noindent 2)$
[FFT(y)]_{i}=[Fy]_i=\sum_{k=0}^{n-1}{y_k(w_n^i)^k} = b_i
$ аналогично\\

\noindent 3)$\Lambda = diag(\lambda_i)$;  $[FFT(x*y)]_i = [Fcirc(x)y]_i = [\Lambda Fy]_i = \lambda_i\cdot[Fy]_i = \lambda_ib_i$

\section*{Задача 6}
Транспонируем равенство $CM_n=M_n\Lambda \Rightarrow M_nC^T=\Lambda M_n$, где 
\[
C^T=
\begin{pmatrix}
c_0&c_1&c_2&\cdots&c_n\\
c_{n}&c_0&c_1&\cdots&c_{n-1}\\
\vdots&\vdots&\vdots&\cdots\vdots\\
c_1&c_2&c_3&\cdots&c_0\\
\end{pmatrix}
\]
В последнем равентсве вычислим первый столбец в каждом произведении соответсвенно:\\
\[M_n(c_0, c_n, c_{n-1}, \cdots, c_1)^T = (\lambda_0, \lambda_1, \cdots, \lambda_n)^T.~~~
F_n=\dfrac{1}{\sqrt{n+1}}M_n
\]
В силу того, что собственные вектора матрицы сохраняются при ее транспонировании, то утверждение доказано.\\

С помощью БПФ вычисляем:
$
\begin{pmatrix}
1&1&1&1\\
1&i&-1&-i\\
1&-1&1&-1\\
1&-i&-1&i\\
\end{pmatrix}
\begin{pmatrix}
1\\2\\4\\6\\
\end{pmatrix}
\begin{pmatrix}
13\\-3-4i\\-3\\-3+4i\\
\end{pmatrix}
$
\\
\section*{Задача 7}
Для каждого элемента $i \in A ~a_i=1$, иначе $a_i = 0$. \\
Умножим за $O(m\log{m})~(a_1x+\cdots+a_mx^m)(a_1x+\cdots+a_mx^m) = \sum_{k=2}^{2m}{p_kx^k}$. Если коэффициент $p_k$ ненулевой, то $k \in A+A$, т.~к. $p_k \neq 0 \Rightarrow \exists a_i\neq 0, a_j\neq 0: i+j=k$, т.~е. действительно $k \in A$.\\

\section*{Задача 9}
За $O(n\log{n})$ по ДПФ найдем $\{y_k\}_{k=0}^{n-1}$, а сумму мнимых и действительных частей считаем за линейное время (один проход по вектору). В итоге, решили за $O(n\log{n})=o(n^2)$.



\end{document}