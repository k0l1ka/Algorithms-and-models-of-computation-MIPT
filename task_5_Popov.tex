\documentclass[a4paper,12pt]{article} % тип документа

% report, book

%  Русский язык

\usepackage[T2A]{fontenc}			% кодировка
\usepackage[utf8]{inputenc}			% кодировка исходного текста
\usepackage[english,russian]{babel}	% локализация и переносы


\usepackage{xcolor}
\usepackage{hyperref}
 
 % Цвета для гиперссылок
\definecolor{linkcolor}{HTML}{799B03} % цвет ссылок
\definecolor{urlcolor}{HTML}{799B03} % цвет гиперссылок
 
\hypersetup{pdfstartview=FitH,  linkcolor=linkcolor,urlcolor=urlcolor, colorlinks=true}

% Математика
\usepackage{amsmath,amsfonts,amssymb,amsthm,mathtools} 


\usepackage{wasysym}
% мои добьавки-----------------------------------------
\usepackage{ dsfont }
\usepackage{listings}

%мое----------------------------------------------------
%Заговолок
\author{Попов Николай}
\title{Домашнее задание 1\LaTeX{}}

%--------------------
\usepackage{tikz}

\begin{document}

\section*{Задача 1}

Рассмотрим варианты того как можно начать заполение полосы с одного ее края. Либо ложим одну вертикальную табличку, либо две горизонталные, либо квадрат (все плотно к краю). Получаем рекуренту на количество способов $F_n = 3F_{n-2} + F_{n-1}$. Решаем её:$~ \Lambda^2 -\Lambda -3 = 0 \Rightarrow \Lambda~= 0.5(1 \pm \sqrt{13}),~ F_n = C_1(0.5(1+\sqrt{13}))^n + C_2(0.5(1-\sqrt{13}))^n $. \\

Из того, что $ F_1 = 1, F_2 = 4$ получаем $C_1 = \frac{1}{2\sqrt{13}}(1+\sqrt{13}), C_1 = \frac{-1}{2\sqrt{13}}(1-\sqrt{13})$. В итоге, $F_n = \frac{1}{\sqrt{13}}\left(\frac{1+\sqrt{13}}{2}\right)^{n+1} - \frac{1}{\sqrt{13}}\left(\frac{1-\sqrt{13}}{2}\right)^{n+1} 
$\\
Переводим линейную рекуренту для $F_n$ в систему рекурент\\
$\begin{pmatrix}
F_{n+1}\\
F_{n}\\
\end{pmatrix}
= \begin{pmatrix}
1&3\\
1&0\\
\end{pmatrix}\begin{pmatrix}
F_{n}\\
F_{n-1}\\
\end{pmatrix} \Rightarrow \begin{pmatrix}
F_{n+1}\\
F_{n}\\
\end{pmatrix} 
= \begin{pmatrix}
1&3\\
1&0\\
\end{pmatrix}^{n-1}
\begin{pmatrix}
F_{2}\\
F_{1}\\
\end{pmatrix}$
Считаем первые несколько степеней матрицы по модулю 31:
$
\begin{pmatrix}
1&3\\
1&0\\
\end{pmatrix}^{2} 
= \begin{pmatrix}
4&3\\
1&3\\
\end{pmatrix}$\\
$
\begin{pmatrix}
1&3\\
1&0\\
\end{pmatrix}^{3} = \begin{pmatrix}
7&12\\
4&3\\
\end{pmatrix},
\begin{pmatrix}
1&3\\
1&0\\
\end{pmatrix}^{4} = \begin{pmatrix}
19&21\\
7&12\\
\end{pmatrix},
\begin{pmatrix}
1&3\\
1&0\\
\end{pmatrix}^{5} = \begin{pmatrix}
40&57\\
19&21\\
\end{pmatrix}=
\begin{pmatrix}
9&26\\
19&21\\
\end{pmatrix}
$\\
$\begin{pmatrix}
1&3\\
1&0\\
\end{pmatrix}^{6} = \begin{pmatrix}
97&120\\
40&57\\
\end{pmatrix}=
\begin{pmatrix}
4&27\\
9&26\\
\end{pmatrix}
$,
$\begin{pmatrix}
1&3\\
1&0\\
\end{pmatrix}^{7} = \begin{pmatrix}
217&291\\
97&120\\
\end{pmatrix}=
\begin{pmatrix}
0&12\\
4&27\\
\end{pmatrix}
$
$\begin{pmatrix}
1&3\\
1&0\\
\end{pmatrix}^{8} = \begin{pmatrix}
508&651\\
217&291\\
\end{pmatrix}=
\begin{pmatrix}
12&0\\
0&12\\
\end{pmatrix} = 12 E_2
$. Тогда 

$\begin{pmatrix}
F_{30000+1}\\
F_{30000}\\
\end{pmatrix}
= 12^{30000/8}E_2
\begin{pmatrix}
F_{2}\\
F_{1}\\
\end{pmatrix} = 12^{3750}
\begin{pmatrix}
4\\
1\\
\end{pmatrix} = \left(12^{125}\right)^{30} \begin{pmatrix}
4\\
1\\
\end{pmatrix} \underset{31}\equiv \begin{pmatrix}
4\\
1\\
\end{pmatrix}$\\

 Т.~е. $F_{30000}\underset{31}\equiv 1$
 
 \section*{Задача 2}
$(i)~ g(2) = 4 + 2 + 2 + 4 = 12 $ - количество путей длины два, в которых на втором месте после $1$ идут вершины $1$, $2$, $3$ и $4$.\\
 Общая формула для $g(n)$~---~сумма элементов первой строки в  матрице $A^n$, где $A$~--~матрица смежности графа. Формула верна, т.к. $A^n[i][j]$ есть количество путей между вершинами $i$ и $j$, состоящих ровно из $n$ рёбер. Покажем это по индукции:\\
 1. матрица смежности $A$ является ответом на задачу при k=1~---~она содержит количества путей длины 1 между каждой парой вершин. \\
 2. В предположении, что $A^{k}$ содержит в каждой ячейке число путей длины $k$ между вершинами $i$ и $j$, для количества путей длины $k+1$ между этими вершинами получаем формулу $\sum_{q=1}^{n}{A^k[i][q]\cdot A[q][j]}$ ($n$ вершин в графе), что равно $A^{k+1}[i][j]$. \\
 Поэтому сложив элементы первой строки в $A^n$, получим число путей длины $n$ из первой вершины до всех вершин в графе, включая её саму.\\
 $(ii)$~Найдем рекуренту ввиду доказанного выше факта:\\
 Пусть $A^n = \begin{pmatrix}
 a_{11}&a_{12}&a_{13}&a_{14}&\\
 a_{21}&a_{22}&a_{23}&a_{24}&\\
 a_{31}&a_{32}&a_{33}&a_{34}&\\
 a_{41}&a_{42}&a_{43}&a_{44}&\\
 \end{pmatrix}$. Тогда $g(n) = a_{11}+a_{12}+a_{13}+a_{14}$\\
 
 
 $A^{n+1} = A^n\begin{pmatrix}
 1&1&1&1\\
 1&0&0&1\\
 1&0&0&1\\
 1&1&1&1\\
 \end{pmatrix} = \begin{pmatrix}
 g(n)&a_{11} + a_{14}& a_{11} + a_{14}&g(n)&\\
 b_{21}&b_{22}&b_{23}&b_{24}&\\
 b_{31}&b_{32}&b_{33}&b_{34}&\\
 b_{41}&b_{42}&b_{43}&b_{44}&\\
 \end{pmatrix}$\\
 
  $g(n+1) = 2g(n) + 2(a_{11}+a_{14})$
 \\
 
 $A^{n+2} = A^{n+1}\begin{pmatrix}
 1&1&1&1\\
 1&0&0&1\\
 1&0&0&1\\
 1&1&1&1\\
 \end{pmatrix} = \begin{pmatrix}
 g(n+1)&2g(n)&2g(n)&g(n+1)&\\
 b_{21}&b_{22}&b_{23}&b_{24}&\\
 b_{31}&b_{32}&b_{33}&b_{34}&\\
 b_{41}&b_{42}&b_{43}&b_{44}&\\
 \end{pmatrix}$\\
 
 
 Ответ:
 $g(n+2) = 2g(n+1)+ 4g(n)$
 
$(iii)~$ При непосредственном вычислении по рекуррентной формуле $g_{20000}~mod~29$ будем вычислять новое значение $g(n)~mod~29$, зная предыдущие два значения $g(n-1)~mod~29$ и $g(n-2)~mod~29$. При этом все значения будем производить по модулю $29$. При этом на каждом из $n-2$ шагов (вычисления начинаем от $g_1 = 4$ и $g_2 = 12$) все операции будем производить по модулю, т.е. будем работать с числами от $0$ до $28$, а значит, арифметические операции с ними (умножение, сложение и приведение по модулю) выполняются за константное время с использованем константной памяти. В итоге, трудоемкость вычисления $g(n)$ по любому модулю есть $\Theta(n)$. При $n = 20 000$ получаем порядка $20 000$ итераций, причем на каждой из них затрачивается времени не более $2\log^2 {29}+2\log{4 \cdot 29} + \log^2{6\cdot 29}$ (это соответственно умножение, сложение и поиск остатка по модулю).\\
$(iv)~$ Отсутствие периода означает, что каждое новое число уникально в последовательности. Каждое число в последовательности считается по двум предыдущим, т.е. если периода нет, то пары остатков, которые дают соседние элементы в последовательности, всегда получаются новые. По модулю $m$ таких пар не более чем $m^2$, а значит, при $n > m^2$, какая-то пара встретится больше раза. А значит, все последующие элементы тоже повторятся при вычислении по формуле. Длина периода по модулю $m$ не более $m^2$, значит, его точно найдем, вычислив не более $m^2$ пар за общее время $O(m^2 \log^3{m})$. При нахождении совпадения (проверку совпадения можно быстро проверять по матрице $m\times m$ с отметками наличия таковой пары) знаем длину периода $p$. Тогда $ (g(n)~;~g(n-1)) = (g(n~mod~p)~;~g((n-1)~mod~p)$, что вычислим за $O(\log^3{n})$. \\

Д-1)~Аналитическая формула $g(n)= \frac{1}{\sqrt{5}}(1+\sqrt{5})^{n+1} - \frac{1}{\sqrt{5}}(1-\sqrt{5})^{n+1}$ (рекурента как в контрольной и $g(1) = 4, g(2) = 12 $,отсюда $C_1 =\frac{1}{\sqrt{5}}(1+\sqrt{5})$ и $C_1 =\frac{-1}{\sqrt{5}}(1-\sqrt{5})$). Обоснование алгоритма в том, что все вычисления, включая извлечение корня и нахождение обратного, можно производить в группе остатков по модулю. Т.~к. $11^2 \underset{29} \equiv 5$, то$~g(20000) \underset{29} \equiv 11^{-1}(1+11)^{20001}-11^{-1}(1-11)^{20001}\underset{29} \equiv 8\cdot 12^{20001~mod~28}- 8\cdot (-10)^{20001~mod~28} \underset{29} \equiv 8(12^9+10^9) \underset{29} \equiv 8$. Т.к. квадратичный корень по модулю известен (на случай если в формуле есть извлечение корня), то необходимо произвести операции по модулю $m$: нахождение обратного, сложение, нахождение остатка и умножение с временем работы соответсвенно $\log{m}, \log{m}, \log^2{m},\log^2{m}$. Умножение придется повторить не более $\log{m}$ раз (при бинарном возведении в степень). В итоге асимптотика $\log^3{m}$.\\
   
   \section*{Задача 4}
$(1)~L \in NP\text{-}c \Leftrightarrow \forall L' \in NP \hookrightarrow L' \leq_p L \wedge L \in NP \Leftrightarrow \overset{-}{L'} \leq_p \overset{-}L \wedge L \in NP$ \\
$(2)~L \in co-NP \Leftrightarrow \overset{-}L \in NP$\\
$(1)\wedge(2) \Leftrightarrow \forall L' \in NP \hookrightarrow \overset{-}{L'} \in NP \Rightarrow L' \in co\text{-}NP$\\
Отсюда $NP = co\text{-}NP$

   \section*{Задача 6}
1)Выберем позиции для 5 единиц (орлы), остальные 5 позиций заполним нулями (решки). Вероятность есть отношение числа  благоприятствующих событий к числу всех событий (для всех пунктов). Тогда $P~=~\frac{C^{5}_{10}}{2^{10}}~=~\frac{63}{256}$\\
2)Аналогично, $P=\sum_{k=6}^{10}{\frac{C_{10}^k}{2^{10}}} = frac{193}{512}$\\
3)Выбор первых пяти символов определяет всю десятку: $P=\frac{2^5}{2^{10}}=\frac{1}{32}$\\
4)Считаем вероятность противоположного события, когда орлов выпало от нуля до трех раз. Пусть у нас есть длинная цепочка, начнем расставлять на первые позиции $1$ и $0$ так, чтобы единиц подряд не было 4 и более. Вариантов начал разных 4:$0\cdots, 10\cdots, 110\cdots, 1110\cdots$. Многоточия рассматриваются также, так как перед ней ноль. Складывая предыдущие четыре числа продолжаем рекурентный ряд: $2~4~8~15~29~56~108~208~401~773 
$. Тогда вероятность исходного события равна $\frac{2^{10}-773}{2^{10}} = \frac{251}{1024}$\\
\section*{Задача 7}
$(i)~B = \text{сумма равна 7}$
$A = \text{на первой кости 6}$
\[P(B) = \dfrac{6}{6\cdot6}= \dfrac{1}{6}\]
\[P(A|B) = \frac{P(A\cap B)}{B} = \frac{P(B|A)P(A)}{P(B)} = \dfrac{1/6\cdot1/6}{1/6} = \dfrac{1}{6}\]
$(ii)~E(max\{x_1,x_2\})+E(min\{x_1,x_2\}) = E(max\{x_1,x_2\}  + min\{x_1,x_2\}) = $ $ = E\left(\dfrac{x_1+x_2}{2}+\dfrac{|x_1-x_2|}{2})+\dfrac{x_1+x_2}{2}-\dfrac{|x_1-x_2|}{2}\right) = E\left(x_1+x_2\right) =$ \\$=  \sum_{k=2}^{12}{kP(x_1+x_2=k)} = \dfrac{2}{36}+\dfrac{3\cdot2}{36}+\dfrac{4\cdot3}{36}+\dfrac{5\cdot4}{36}+\dfrac{6\cdot5}{36}+\dfrac{6\cdot7}{36}+\dfrac{8\cdot5}{36}+\dfrac{9\cdot4}{36}+\dfrac{10\cdot3}{36}+\dfrac{11\cdot2}{36}+\dfrac{12}{36} = \dfrac{58}{9} = 6\dfrac{4}{9} $\\

$(iv)~$Независимость равносильна $P(A\cap B) = P(A)\cdot P(B)$\\
A-четное число, B - кратное трем, $P(A)=\dfrac{1}{2}, P(B)=\dfrac{1}{3}, P(A\cap B) = $ $ = P(\text{выпало 6}) = \dfrac{1}{6}$. Т.~е. события независимы.\\

$v~$ На $n$ вершинах может быть $C_{n}^{2}$ возможных ребер. Тогда возможных графов $2^{C_{n}^{2}}= 2^{\frac{n(n-1)}{2}}$. При этом на $n$ вершинах простых циклов может быть $\frac{n!}{2n}$, т.~к. каждая перестановка соответствует простому пути, но $n$ сдвигов перестановки и инверсия каждой задают один путь. Тогда вероятность того, что случайный граф - простой цикл с учетом формулы Стирлинга при больших $n, n!\sim \sqrt{2\pi n}n^{n}e^{-n}$ 

\[\dfrac{(n-1)!}{2^{0.5n(n-1)+1}}\sim \dfrac{\sqrt{2\pi (n-1)}(n-1)^{(n-1)}e^{-n+1}}{2^{0.5n(n-1)+1}}\sim e^{-n+1-0.5n(n-1)\ln{2}+(n-1)\ln{(n-1)}+0.5\ln(2\pi(n-1))}\]

Что стремится к нулю при $n\rightarrow \infty$\\

\section*{Задача 8}

Пусть в первой урне $a$ белых и  $b$ черных шаров, а во второй $x$ белых и  $y$ черных. Тогда $a+b = x+y$
A = (взятые из первой урны $n$ шаров белые) и B = (взятые $n$ шаров из второй урны все либо белые, либо все черные).
\[P(A) = \left(\dfrac{a}{a+b}\right)^n, P(B) = \left(\dfrac{x}{x+y}\right)^n+\left(\dfrac{y}{x+y}\right)^n \Leftrightarrow x^n+y^n =a^n (a \leq x+y)\]  
По теореме Ферма при $n \geq 3$ это уравнение не имеет натуральных решений. Поэтому единственным решением являются два случая: в первой урне все шары белые, а во второй либо все черные, либо все белые.\\

\section*{Задача 9}
Рассмотрим две комбинации нулей и единиц длины $n+1$: $<\cdots110>$ и $<\cdots101>$. Найдем вероятность того, что среди первых $n$ символов последовательности $<\cdots110>$ нет тройки $110$, а при дописывании еще одного нуля она появляется (событие А) и того же для тройки $101$ при дописывании $1$ (событие В). Тогда \[
P(A) = 1 - \frac{1}{2^{n+1}}\sum_{i=1}^{[n/3]}{K_1(i)}
\]
Где $K_1(i)$ есть количество комбинаций с $i$ тройками $110$).
Аналогично, 
\[
P(B) = 1 - \frac{1}{2^{n+1}}\sum_{i=1}^{[n/2]}{K_2(i)}
\]

Кроме того, что во второй сумме элементов больше, также верно, что $K_1(i) \leq K_2(i)~\forall i \in N$. Тогда $P(A) > P(B)$, значит, тройка $101$ встретится раньше с большей вероятностью.\\

\section*{Задача 10}  
$(i~)$Т.к. с помощью генератора получаем пары $00, 10, 01, 11$ с равными вероятностями $\dfrac{1}{4}$,то чтобы получить нуль, запустим дважды генератор и, если получим $00$, то вернем $0$ (получается вероятность $1/3$), при $11$ не возвращаем ничего и повторяем процедуру, иначе вернем $1$ (вероятность $2/3$). В худшем случае алгоритм ничего не будет выдавать, а в лучшем вернет сразу нужную цифру с  нужной вероятностью (при печатании подряд $n$ символов их соотношение будет стремится к $1:2$ при $n \rightarrow \infty$ ) \\
$(ii)~$Чтобы генерировать $0$ и $1$ с равной вероятностью, заметим что c  равной вероятностью $\frac{2}{9}$ в новом генераторе получаются пары $10$ и $01$ (пусть они соответсвуют $1 $ и $ 0$). Тогда дважды запустив новый генератор и получив $00$ и $11$ ничего не возвращаем и повторяем запуск, иначе в зависимости от результата вернем $1 $ или $ 0$.\\

\end{document}