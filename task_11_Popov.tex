\documentclass[a4paper,12pt]{article} % тип документа

% report, book

%  Русский язык

\usepackage[T2A]{fontenc}			% кодировка
\usepackage[utf8]{inputenc}			% кодировка исходного текста
\usepackage[english,russian]{babel}	% локализация и переносы

\usepackage[margin=1in]{geometry}
\usepackage{xcolor}
\usepackage{hyperref}
 
 % Цвета для гиперссылок
\definecolor{linkcolor}{HTML}{799B03} % цвет ссылок
\definecolor{urlcolor}{HTML}{799B03} % цвет гиперссылок
 
\hypersetup{pdfstartview=FitH,  linkcolor=linkcolor,urlcolor=urlcolor, colorlinks=true}

% Математика
\usepackage{amsmath,amsfonts,amssymb,amsthm,mathtools} 


\usepackage{wasysym}
% мои добьавки-----------------------------------------
\usepackage{ dsfont }
\usepackage{listings}

%мое----------------------------------------------------
%Заговолок
\author{Попов Николай}
\title{Домашнее задание 1\LaTeX{}}

%--------------------
\usepackage{tikz}

\begin{document}

\section*{Задача 1}

$
x=41; ~y = x^e~ mod~ N  = 41^3 ~mod~391 = 105
$\\
В открытом доступе $y = 105,~ N = 391,~ e = 3$.\\
Зная $p$ и $q$, получаю 
$
d = e^{-1} ~mod~ (p-1)(q-1) = 3^{-1}~mod~352 = 235
$\\
Затем дешифрую $x = y^d~mod~N = 105^{235}~mod~391 = 41 $

\section*{Задача 2}

Злоумышленнику известны $N, e, d$. Он находит $p,q : N = pq$. Нахождение обратного к $3$ по модулю $M = (p-1)(q-1)$ из уравнения $3d+Mb = 1$, где $b \in Z$, возможно по двум схемам в зависимости от остатка $M$ при делении на $3$:\\

1)Если остаток при делении $M$ на 3 равен 1:
\[3d+Mb = 1\]
\[
3\begin{pmatrix}
1\\0\\
\end{pmatrix}\hspace{30pt}
M\begin{pmatrix}
0\\1\\
\end{pmatrix}
\]   

\[
3\begin{pmatrix}
1\\0\\
\end{pmatrix}\hspace{30pt}
1\begin{pmatrix}
-\left\lfloor\frac{M}{3}\right\rfloor\\1\\
\end{pmatrix}
\]  
$d = -\left\lfloor\frac{M}{3}\right\rfloor, b = 1;~~~ M=-3d+1$\\

2)Если остаток при делении $M$ на 3 равен 2:
\[3d+Mb = 1\]
\[
3\begin{pmatrix}
1\\0\\
\end{pmatrix}\hspace{30pt}
M\begin{pmatrix}
0\\1\\
\end{pmatrix}
\]   

\[
3\begin{pmatrix}
1\\0\\
\end{pmatrix}\hspace{30pt}
2\begin{pmatrix}
-\left\lfloor\frac{M}{3}\right\rfloor\\1\\
\end{pmatrix}
\]   


\[
1\begin{pmatrix}
1+\left\lfloor\frac{M}{3}\right\rfloor\\-1\\
\end{pmatrix}\hspace{30pt}
2\begin{pmatrix}
-\left\lfloor\frac{M}{3}\right\rfloor\\1\\
\end{pmatrix}
\]   

$d = 1+\left\lfloor\frac{M}{3}\right\rfloor, b = -1;~~~ M=3d-1$. \\

Поскольку $N-M+1=p+q=s$
~(обозначим так) и $N=pq$, то получаем: $p=s-q$ и $N=(s-q)q \Rightarrow q^2-sq+N=0$. Отсюда $q = 0.5(s+\sqrt{s^2-4N})$. Для вычисления $M$ есть два варианта, но правильным будет только один, поскольку $M$ не может иметь два разных остатка. Поэтому посчитаем для обоих $M$ $p$ и $q$. В одном из вариантов получатся простые числа, что и будет ответом.\\
        


\section*{Задача 3}
$ N = 2021, e = 25, M =(p-1)(q-1)=42\cdot46=1932, p=43, q = 47$: Найдем обратный к $e$ по модулю $M$.
\[
25d+1932b = 1
\]
\[
25\begin{pmatrix}
1\\0\\
\end{pmatrix}\hspace{30pt}
1932\begin{pmatrix}
0\\1\\
\end{pmatrix}
\]
\[
25\begin{pmatrix}
1\\0\\
\end{pmatrix}\hspace{30pt}
7\begin{pmatrix}
-77\\1\\
\end{pmatrix}
\]
\[
4\begin{pmatrix}
232\\-3\\
\end{pmatrix}\hspace{30pt}
7\begin{pmatrix}
-77\\1\\
\end{pmatrix}
\]
\[
4\begin{pmatrix}
232\\-3\\
\end{pmatrix}\hspace{30pt}
3\begin{pmatrix}
-309\\4\\
\end{pmatrix}
\]
\[
1\begin{pmatrix}
541\\-7\\
\end{pmatrix}\hspace{30pt}
3\begin{pmatrix}
-309\\4\\
\end{pmatrix}
\]
Получаем $d=541$~---~степень, в которую надо возвести сообщение для электронной подписи.\\


\section*{Задача 4}
a)Используем шаги бинарного поиска с одним сравнением выбранного элемента с элементом правее него на каждом шаге. На каждом шаге выбираем половину, в направлении которой значение растет (если элементы равны, то выбор произвольный). В этой половине действительно будет горка, поскольку, если значения продолжат расти, то горка будет в конце, если начнут убывать, то горка будет внутри отрезка, как и если значение постоянно с некоторого момента до конца отрезка. Алгоритм закончится, поскольку длина рассматриваемых отрезков убывает вдвое на каждом шаге. Шагов будет $\lfloor\log_2{n}\rfloor$, на каждом из которых ровно одно сравнение.\\

б)В любом массиве есть горка. Начиная просмотр с каждого конца массива получаем, что горки нет, если значения сторого возрастают при движении к середине массива. Но тогда в середине обязательно будет горка. Поэтому для любого массива движение в сторону увеличения значения элементов приведет нас к горке. Попытаемся как можно быстрее двигаться в сторону роста значений массива. Для этого будем уменьшать длину рассматриваемого участка значений, где точно есть горка. Если соотношение частей при выборе элемента не равно  $1:1$, то в худшем случае будет выбираться большая часть, что приведет к большему числу сравнений. При выборе среднего элемента получаем минимальное число сравнений, поскольку при этом длина рассматриваемого отрезка убывает быстрее всего.\\


\section*{Задача 5}
 1)Клика переходит в независимое множество в реберном дополнении графа.\\
 
 2)Наличие двух клик на дизьюнктых подмножествах вершин в графе означает наличие двух долей в реберном дополнении графа.\\
 
 3)Двудольность реберного дополнения графа равносильна условию задачи.\\
  Проверка двудольности производится за полином времени: начинаем $BFS$ из произвольной вершины, окрасив ее в цвет $1$. Всех ее соседей красим в цвет $0$ и так далее, чтобы цвета соседей были разные. Если в процессе обхода найдется пара соседних вершин с одинаковыми цветами, то ответ <<Нет>>, иначе ~---~ <<Да>>.\\

 Значит, данный язык не является $NPc$.
 
 \section*{Задача 6}
Решим данную задачу за полином времени. Возвращение и повторный проход по ребрам увеличивает длину обычного пути на четное число. Если есть пути длины $10$ и $11$, то обратными проходами можно увеличить их длину до любого $S\geq 10$. Проверим наличие путей длины $10$ и $11$ между $s$ и $t$ возведением матрицы смежностей в эти степени. В итоге, задача решается за $O(|V|^3)$. Значит, она принадлежит $co$-$NP$, поскольку принадлежит $P$.\\

\section*{Задача 7} 

Отсортируем ребра по убыванию их веса. Далее будем добавлять вершины ребер в множества двух цветов (в лучшем случае это две доли) , начиная   
 с ребер большего веса, так, что их вершины имеют по возможности разные цвета. Делаем это, пока каждая из вершин не попадет в какое-то множество.\\
  При этом получаем оптимальную для задачи раскраску графа, поскольку при таком алгоритме ребра с большими весами исключаются из рассмотрения наилучшим образом, и вследствие этого, максимальный вес ребра с одноцветными концами получается минимальным по всем возможным раскраскам.\\   
 Сортировка работает за $O(m\log{m})$.
\end{document}