\documentclass[a4paper,12pt]{article} % тип документа

% report, book

%  Русский язык

\usepackage[T2A]{fontenc}			% кодировка
\usepackage[utf8]{inputenc}			% кодировка исходного текста
\usepackage[english,russian]{babel}	% локализация и переносы

\usepackage[margin=1in]{geometry}
\usepackage{xcolor}
\usepackage{hyperref}
 
 % Цвета для гиперссылок
\definecolor{linkcolor}{HTML}{799B03} % цвет ссылок
\definecolor{urlcolor}{HTML}{799B03} % цвет гиперссылок
 
\hypersetup{pdfstartview=FitH,  linkcolor=linkcolor,urlcolor=urlcolor, colorlinks=true}

% Математика
\usepackage{amsmath,amsfonts,amssymb,amsthm,mathtools} 


\usepackage{wasysym}
% мои добьавки-----------------------------------------
\usepackage{ dsfont }
\usepackage{listings}

%мое----------------------------------------------------
%Заговолок
\author{Попов Николай}
\title{Домашнее задание 1\LaTeX{}}
\usepackage{float}
%--------------------
\usepackage{tikz}
%% Useful packages
\usepackage{amsmath}
\graphicspath{{Pictures/}} 
\DeclareGraphicsExtensions{.pdf,.png,.jpg}


\begin{document}

\section*{Задача 1}
В исходной задаче обозначим: $x = x_1, y = x_2$. \\
\[
y_1(2x_1+x_2)+y_2(x_1+3x_2)\leq 3y_1+5y_2
\]
\[
x_1(2y_1+y_2)+x_2(y_1+3y_2)\leq 3y_1+5y_2 = f
\]

Двойственная задача:
$
\begin{vmatrix}
3y_1 + 5y_2 \rightarrow min\\
2y_1+y_2=1\\
y_1+3y_2=1\\
y_1 \geq 0\\
y_2 \geq 0\\
\end{vmatrix}
$\\

Решаем двойственную задачу:
$f= 3y_1 + 5y_2 = (3y_1 + 4y_2)+y_2 = 2 +y_2 \Rightarrow f_{min}=2$.
В силу того, что функция $f$ является ограничением сверху на искомое выражение в прямой задаче, то если найдем минимальное её значение, то для интересующей нас функции $g=x_1+x_2$ будет найдена минимальная верхняя граница. Т.~к. для значения $d = max\{g\}$ из ограничений прямой задачи можно вывести $g \leq d$, а минималььным $d : g \leq d$ является $f_{min}$, то $g_{max}=f_{min}$.\\
Отсюда оптимальное значение функции в прямой задаче равно $2$.\\

\section*{Задача 2}
Каждому ребру присвоим индивидуальную переменнную $x_j$, равную потоку через это ребро. Если заданы пропускные способности ребер $c_j$, то запишем неравенства $x_j\leq c_j$. Для каждой вершины $v$, с  множеством входящих ребер $In$ и множеством выходящих ребер $Out$ запишем по одному неравенству $\epsilon_v\cdot\sum_{k \in In}{x_k}= \sum_{i \in Out}{x_i}$. Пусть $s$ является истоком и $t$~---~стоком, ребра $S$ выходят из истока, ребра $T$ входят в сток. Тогда добавим к ограничениям уравнение $\sum_{k \in S}{x_k}=\sum_{i \in T}{x_i}$. В полученной задаче линейного программирования, решаемой за полином времени, нужно максимизировать функцию   $f = \sum_{i \in T}{x_i}$. \\

\section*{Задача 3}
Нам необходимо минимизировать значение величины $f = max|ax_i+by_i+c| ~\forall i:1\leq i \leq 7$. Решим систему:
\[
\begin{vmatrix}
f \rightarrow min\\
\forall~k(1\leq k \leq 7)~ax_k +by_k  +c \leq f\\
\forall~k(1\leq k \leq 7)~-f\leq ax_k +by_k  +c\\
\end{vmatrix}\Leftrightarrow
\begin{vmatrix}
f \rightarrow min\\
ax_k +by_k  +c-f \leq 0\\
ax_k +by_k  +c + f \geq 0	\\
\end{vmatrix}\]

Значит найдем значения переменных $a, b, c, f$, доставляющие минимум $f$, которые и будут ответом.
Для указанной сиситемы точек вспомогательная ЛП:
\[\begin{vmatrix}
f \rightarrow min\\
a + 3b  +c-f \leq 0\\
2a + 5b  +c-f \leq 0\\
3a + 7b  +c-f \leq 0\\
5a + 11b  +c-f \leq 0\\
7a + 14b  +c-f \leq 0\\
8a + 15b  +c-f \leq 0\\
10a + 19b  +c-f \leq 0\\
a + 3b  +c+f \geq 0\\
2a + 5b  +c+f \geq 0\\
3a + 7b  +c+f \geq 0\\
5a + 11b  +c+f \geq 0\\
7a + 14b  +c+f \geq 0\\
8a + 15b  +c+f \geq 0\\
10a + 19b  +c+f \geq 0\\
\end{vmatrix}\]
 

\section*{Задача 4}
Путь через все вершины с увеличением $x_3$ указан на чертеже номерами вершин в порядке их обхода. Из чертежа получаем очевидный факт, что порядок обхода вершин не зависит от $\varepsilon$, потому что $0 < \varepsilon < 0.5$. Т.~е. для любого $\varepsilon$ есть такой путь.\\

\begin{figure}[H]
\center{\includegraphics[scale=0.163]{t11_p4.png}} 
\caption{Фигура при $\varepsilon = \frac{2}{7}$ из задачи №4} 
\end{figure}



\section*{Задача 5}

Доказательство в две стороны:\\
1) Пусть система 
$\begin{vmatrix}
Ax \leq b\\
x \geq 0\\
\end{vmatrix}
$ совместна. Покажем, что тогда система 
$\begin{vmatrix}
A^{T}y \geq 0\\
y \geq 0\\
b^{T}y < 0
\end{vmatrix}
$ несовместна.\\
Из первой системы получаем $b^{T} \geq x^{T}A^{T}$ и, в силу её совместности, $\exists x$, удовлетворяющий ей.\\ 
Предположим, что вторая система совместна, тогда \\ $\forall y \geq 0: A^{T}y \geq 0 \rightarrow b^{T}y 
\geq x^{T}(A^{T}y) \geq 0 \Rightarrow $ невыполнено последнее неравенство второй системы, значит, она несовместна.\\

(второй вариант того же по сути:)\\
1.1) Пусть совместна вторая система, покажем, что первая система несовместна:\\
$\exists y$, удовлетворяющий второй системе. Предположим, что первая система совместна. Тогда \\
$\exists x\geq 0:<Ax,y>~\leq~ <b,y>=b^{T}y<0$, но $<Ax,y> = <x,A^{T}y> ~<~ 0$ противоречит с тем, что $x \geq 0; A^{T}y
\geq 0$, т.~е. $<x,A^{T}y>~\geq~0$. Тогда первая система не совместна.\\
(тут $<a,b>$ - сумма покомпонентных произведений векторов $a$ и $b$)\\ 
 
 2) Теперь получим из несовместности первой системы совместность второй системы. Из несовместности первой системы имеем $\forall x 
 \geq 0 \hookrightarrow Ax 
 > b$. Очевидно, что $\exists y > 0 : <x,A^{T}y> = <Ax,y> ~>~ <b,y>$ (это верно для любого такого $y$). Если $A^{T}y < 0$, то имеем неверное утверждение $\exists y >0 ~\forall x 
 \geq 0 \hookrightarrow <x,A^{T}y> ~>~ <b,y>$. Это так, поскольку для конкретного $y$ подберем такой $x$, в котором координата, соответствующая какой-то отрицательной координате вектора $A^{T}y$ будет браться все больше и больше, что невозможно, потому что получаемая строгая верхняя оценка на  постоянное число $<b,y>$ не может быть отрицательной и сколь угодно большой по модулю. Тогда $A^{T}\geq 0$. При $x=0 \rightarrow 0= <x,A^{T}y>~>~<b,y>$, т.~е. $b^{T}y < 0$. ($y=0$ не рассматриваем, поскольку он не является решением второй системы). В итоге, показана совместность второй системы.\\
  
 \section*{Задача 6}
 1) Пусть первая система 
 $\begin{vmatrix}
 Ax\leq 0\\
 x>0\\
 \end{vmatrix}$
 совместна.  Тогда верно $x^{T}A^{T}\leq 0$ и для $\forall y  \geq 0 \rightarrow x^{T}A^{T}y \leq 0$. Если $A^{T}y > 0$, то поскольку $x >0$, получаем ложное утверждение: $x^{T}A^{T}y > 0$. Значит, $A^{T}y \leq 0$ и вторая система  $\begin{vmatrix}
 A^{T}y > 0\\
 y \geq 0\\
 \end{vmatrix}$ несовместна.\\
 
 2) Пусть первая система несовместна. Тогда $\forall x > 0 \rightarrow Ax >0 \Rightarrow x^{T}A^{T} > 0 $, тогда для $y >0$ ($y=0$ не удовлетворяет второй системе) верно $x^{T}A^{T}y > 0 $. Если $A^{T}y \leq 0$, то последнее утверждение неверно. Значит, $A^{T}y > 0$ и вторая система совместна.\\
  

\section*{Задача 7}
Доказательством того, что исходная задача не совпадает в точности с двойственной к двойственной является следующий пример. Найдем двойственную задачу к задаче из семинара без использования модификации: \\
\[
\begin{vmatrix}
x_1+2x_2+3x_3 \rightarrow max\\
4x_1+5x_2+6x_3\leq 7 \\
8x_1+9x_2+10x_3\geq 11\\
x_1\leq0\\
x_3\geq 0\\
\end{vmatrix}
\overset{\text{Двойственная}}{\Rightarrow}
\begin{vmatrix}
7y_1-11y_2\rightarrow min\\
4y_1-8y_2+y_3=1\\
5y_1-9y_2=2\\
6y_1=10y_2-y_4=3\\
y_1\geq 0\\	
y2\geq 0\\
y_3\geq 0\\
y_4\leq 0\\
\end{vmatrix}
\]
 Для нахождения двойственной к двойственной нужно ввести семь переменных $x_i$ по одной для каждого ограничения. Тогда в ней будет семь переменнных, а в исходной задаче было три переменные. Поэтому они не совпадают.\\
 
 При указанной на семинаре обработке простейших неравенств при вычислении двойственной к двойственной задаче число переменных получится такое же, как в исходной задаче, поскольку число <<сложных>> неравенств в двойственной задаче равно числу переменных в исходной задаче. \\ 
 

  
\end{document}