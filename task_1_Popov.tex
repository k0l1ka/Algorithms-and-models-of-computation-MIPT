 \documentclass[a4paper,12pt]{article} % тип документа

% report, book

%  Русский язык

\usepackage[T2A]{fontenc}			% кодировка
\usepackage[utf8]{inputenc}			% кодировка исходного текста
\usepackage[english,russian]{babel}	% локализация и переносы


\usepackage{xcolor}
\usepackage{hyperref}
 
 % Цвета для гиперссылок
\definecolor{linkcolor}{HTML}{799B03} % цвет ссылок
\definecolor{urlcolor}{HTML}{799B03} % цвет гиперссылок
 
\hypersetup{pdfstartview=FitH,  linkcolor=linkcolor,urlcolor=urlcolor, colorlinks=true}

% Математика
\usepackage{amsmath,amsfonts,amssymb,amsthm,mathtools} 


\usepackage{wasysym}
% мои добьавки-----------------------------------------
\usepackage{ dsfont }
\usepackage{listings}

%мое----------------------------------------------------
%Заговолок
\author{Попов Николай}
\title{Домашнее задание 1\LaTeX{}}

%--------------------
\usepackage{tikz}












\begin{document}

\section*{Задача 1}

(3)~Пусть $n$~---~четное число, т.е. $n = 2k $. Тогда для диофантова уравнения уравнения из условия число решений равно $ A_n = \lfloor\frac{n}{6}\rfloor - I(n~mod~6 = 0) $, где $I()$ - индикатор, равный $1$, если утверждение в скобках истинно и равный 0, если оно ложно. Такой ответ следует из того, что выбор $y$ однозначно определяет $x$ в уравнении. А число натуральных $y$, удовлетворяющих уравнению, получаем из того, что $y$ - четное. Т.е. если разбить число $n$ на шестёрки, то каждая из них выражается либо через $2x$, либо $3y$, и также учитываем, что $x $ и  $y $ не равны нулю.\\

Аналогично, для нечетных $n = 2k+1$ сведением к предыдущему рассуждению  получаем $ A_n = \lfloor \frac{n - 3}{6}\rfloor - I((n-3)~mod~6 = 0) + 1$\\

Обьединив эти ответы, можно записать для произвольного $ n$: число решений ранво $ A_n = \lfloor \frac{1}{6}(n - 3\cdot I(n~mod~2 = 1))\rfloor + I(n~mod~2 = 1) + I((n - 3\cdot I(n~mod~2 = 1)~mod~6 = 0 ))$  \\

(2)~Из формулы для $A_n$ получаем $A_n = \theta(n)$\\

(1)~Если подставлять в выражение $ 2x+3y$ всеможножные натуральные числа $x $ и $y$ (неравные нулю), то получим все $n$, удовлетворяющие условию. Тогда, т.к. при $n = 0$ задача не рассматривается, получаем
\[
	\sum_{n=1}^{\infty}{A_nx^n} = \sum_{x, y\in N}{x^{2x+3y}}  = \sum_{x=1}^{\infty}{x^{2x}}\sum_{y=1}^{\infty}{x^{3y}} = \dfrac{x^2}{1-x^2}\cdot\dfrac{x^3}{1-x^3} = \dfrac{x^5}{(1-x^2)(1-x^3)}
\]
 

\section*{Задача 2}

(1)~Покажем, что $s_i \leq \frac{2}{3}s_{i-1}$\\
Т.к. $x_i > y_i$ на каждой итерации, то $x_{i-1} = a_{i-1}y_{i-1} + x_{i-1}$~mod~$y_{i-1}$, где $a_{i-1}\geq 1$\\
Преобразованиями получаем
\[
	x_{i}+y_{i} \leq \frac{2}{3}(x_{i-1}+y_{i-1}) \Leftrightarrow y_{i-1} + x_{i-1}\text{~mod~}y_{i-1}\leq \dfrac{2}{3}y_{i-1}+\frac{2}{3}x_{i-1}\Leftrightarrow \]

\[\Leftrightarrow 2x_{i-1}\geq y_{i-1} + 3(x_{i-1}\text{~mod~}y_{i-1})  
\Leftarrow x_{i-1} \geq 3(x_{i-1}\text{~mod~}y_{i-1})\Leftrightarrow\]
\[
	\Leftrightarrow (2a_{i-1}-1)y_{i-1} \geq x_{i-1}\text{~mod~}y_{i-1}
\]
Что верно, т.к. 
$ x_{i-1}\text{~mod~}y_{i-1} < y_{i-1} \text{~и~} (2a_{i-1}-1)\geq 1$\\

(2)~Т.к. $
F_n = F_{n-1}+F_{n-2}
$ и $F_m < F_{m+1}$, то\\
$\gcd(F_{m+2};F_{m+1}) = \gcd(F_{m+1}+F_{m};F_{m+1}) = \gcd(F_{m+1};F_{m}) = \cdots = \gcd(F_2;F_1) = \gcd(1;1) = 1 $

\section*{Задача 3}

При больших $k~u_i \sim \dfrac{k}{4}$\\
\[
\dfrac{k^3}{4^2} < G(k) = 4G\left(\dfrac{k}{4}\right) + \dfrac{k^3}{4^2} = \dfrac{k^3}{4^2}\sum_{k=0}^{\log_{4}{k}}{\dfrac{1}{4^{2k}}}\leq \dfrac{k^3}{4^2} \sum_{k=0}^{\infty}{\dfrac{1}{4^{2k}}}
\] 
Т.к. последняя сумма убывающей геометрической прогрессии это константа, то получаем, что $G(k)=\Theta(k^3)$

\section*{Задача 4}

(1)Количество открывающих и закрывающих скобок в правильной скобочной последовательности равны, значит, $n$~---~ четное.\\

Из условия на префикс получаем, что в любой скобочной последовательности, удовлетворяющей условию, есть внешняя пара скобок. В противном случае последовательность будет состоять по крайней мере из двух не вложенных друг в друга скобочных последовательностей, а значит, существует префикс, не удовлетворяющий условию.\\

Число правильных скобочных последовательностей из $n-2$  скобок равно 
$
\frac{2}{n}\text{C}_{n-2}^{0.5n-1}
$  \\


\section*{Задача 5}
Т.к. $T(n)$~---~ монотонная неотрицательная функция, то 
\[
	(1)~~~~~10\dfrac{n^3}{\log{n}} \leq T(n)=3T\left(\lceil\frac{n}{\sqrt{3}}\rceil-5\right)+10\dfrac{n^3}{\log{n}} \leq 3T\left(\dfrac{n}{\sqrt{3}}\right)+10\dfrac{n^3}{\log{n}}
\]


Для $G(n) = 3G\left(\dfrac{n}{\sqrt{3}}\right)+10\dfrac{n^3}{\log{n}}$ по мастер-теореме (вариант 3) получаем:
\[
	G(n)=aG\left(\dfrac{n}{b}\right)+f(n),~\log_{b}{a} = 2 < c = 2.5 ,~f(n)=\dfrac{10n^3}{\log{n}}
\]
Т.к. \[
\lim_{n\rightarrow\infty}{\frac{f(n)}{n^c}} = 10\lim_{n\rightarrow\infty}{\frac{\sqrt{n}}{\log{n}}} = \lim_{n\rightarrow\infty}{\frac{n}{2\sqrt{n}}} = \infty\]
 то $f(n) = \Omega(n^c)$ и т.к. выполнено условие регулярности:
\[
	\exists p < 1:~~3f\left(\frac{n}{\sqrt{3}}
	\right) = \dfrac{10n^3}{\sqrt{3}\log{\frac{n}{\sqrt{3}}}} \leq p\cdot\dfrac{10n^3}{\log{n}} = pf(n) \text{~~при~~} n \rightarrow \infty~\Leftarrow
\]


\[
\Leftarrow \exists p: 1 > p \geq \dfrac{\log{n}}{\sqrt{3}\log{\frac{n}{\sqrt{3}}}} = \dfrac{1}{\sqrt{3}}\cdot\dfrac{1}{1 - \frac{\log{\sqrt{3}}}{\log{n}}} = \dfrac{1}{\sqrt{3}} \text{~~при~~} n \rightarrow \infty
\]
То получаем, $G(n)=\Theta(f(n)) = \Theta\left( \dfrac{n^3}{\log{n}}\right)$. \\

В силу (1) получаем $T(n) = \Theta\left( \dfrac{n^3}{\log{n}}\right)$.\\

$\href{https://ru.wikipedia.org/wiki/%D0%9E%D1%81%D0%BD%D0%BE%D0%B2%D0%BD%D0%B0%D1%8F_%D1%82%D0%B5%D0%BE%D1%80%D0%B5%D0%BC%D0%B0_%D0%BE_%D1%80%D0%B5%D0%BA%D1%83%D1%80%D1%80%D0%B5%D0%BD%D1%82%D0%BD%D1%8B%D1%85_%D1%81%D0%BE%D0%BE%D1%82%D0%BD%D0%BE%D1%88%D0%B5%D0%BD%D0%B8%D1%8F%D1%85#%D0%92%D0%B0%D1%80%D0%B8%D0%B0%D0%BD%D1%82_3}{Used~Master-theorem~link}$

\section*{Задача 6}

\[
T(n) = T\left(\dfrac{n}{4}\right) + T\left(\frac{3n}{4}\right) + \Theta(n)
\]

Принимая во внимание, что дерево рекурсивных вызовов данной процедуры будет неполным (т.е. с одной стороны аргумент быстрее приблизиться к константе нежели с другой), получаем оценки сверху и снизу, соответствующие высотам полных деревьев при делении аргумента на 4 и на $\dfrac{4}{3}$ соответственно:
\[
	n\sum_{k=0}^{\log_{4}{n}}{\left(\dfrac{1}{4}+\dfrac{3}{4}\right)^k} \leq T(n) = 1 + \left(\dfrac{1}{4}+\dfrac{3}{4}\right) + \left(\dfrac{1}{4}+\dfrac{3}{4}\right)^2 + \cdots \leq n\sum_{k=0}^{\log_{\frac{4}{3}}{n}}{\left(\dfrac{1}{4}+\dfrac{3}{4}\right)^k}
\] 
Отсюда получаем, что $T(n)=\Theta(n\log{n})$

\section*{Задача 7}



\begin{center}
\begin{tikzpicture}[scale=0.2]
\tikzstyle{every node}+=[inner sep=0pt]
\draw [black] (37.9,-5.7) circle (3);
\draw (37.9,-5.7) node {$\text{\small{S(n)}}$};
\draw [black] (25.2,-15.6) circle (3);
\draw (25.2,-15.6) node {$\text{\small{S(n-1)}}$};
\draw [black] (50,-15.6) circle (3);
\draw (50,-15.6) node {$\text{\small{S(n-3)}}$};
\draw [black] (14.9,-27.9) circle (3);
\draw (14.9,-27.9) node {$\text{\small{S(n-2)}}$};
\draw [black] (37.9,-27.9) circle (3);
\draw (37.9,-27.9) node {$\text{\small{2S(n-4)}}$};
\draw [black] (61,-27.2) circle (3);
\draw (61,-27.2) node {$\text{\small{S(n-6)}}$};
\draw [black] (6,-40) circle (3);
\draw (6,-40) node {$\text{\small{S(n-3)}}$};
\draw [black] (25.2,-40) circle (3);
\draw (25.2,-40) node {$\text{\small{3S(n-5)}}$};
\draw [black] (48,-40.7) circle (3);
\draw (48,-40.7) node {$\text{\small{3S(n-7)}}$};
\draw [black] (69.7,-40) circle (3);
\draw (69.7,-40) node {$\text{\small{S(n-9)}}$};
\draw [black] (40.22,-7.6) -- (47.68,-13.7);
\fill [black] (47.68,-13.7) -- (47.38,-12.81) -- (46.74,-13.58);
\draw [black] (35.53,-7.54) -- (27.57,-13.76);
\fill [black] (27.57,-13.76) -- (28.5,-13.66) -- (27.89,-12.87);
\draw [black] (52.06,-17.78) -- (58.94,-25.02);
\fill [black] (58.94,-25.02) -- (58.75,-24.1) -- (58.02,-24.79);
\draw [black] (47.9,-17.74) -- (40,-25.76);
\fill [black] (40,-25.76) -- (40.92,-25.54) -- (40.21,-24.84);
\draw [black] (27.35,-17.69) -- (35.75,-25.81);
\fill [black] (35.75,-25.81) -- (35.52,-24.9) -- (34.82,-25.62);
\draw [black] (23.27,-17.9) -- (16.83,-25.6);
\fill [black] (16.83,-25.6) -- (17.72,-25.31) -- (16.96,-24.67);
\draw [black] (13.12,-30.32) -- (7.78,-37.58);
\fill [black] (7.78,-37.58) -- (8.65,-37.24) -- (7.85,-36.64);
\draw [black] (16.84,-30.18) -- (23.26,-37.72);
\fill [black] (23.26,-37.72) -- (23.12,-36.78) -- (22.36,-37.43);
\draw [black] (35.73,-29.97) -- (27.37,-37.93);
\fill [black] (27.37,-37.93) -- (28.3,-37.74) -- (27.61,-37.02);
\draw [black] (39.76,-30.26) -- (46.14,-38.34);
\fill [black] (46.14,-38.34) -- (46.04,-37.41) -- (45.25,-38.03);
\draw [black] (58.92,-29.36) -- (50.08,-38.54);
\fill [black] (50.08,-38.54) -- (51,-38.31) -- (50.28,-37.62);
\draw [black] (62.69,-29.68) -- (68.01,-37.52);
\fill [black] (68.01,-37.52) -- (67.98,-36.58) -- (67.15,-37.14);
\end{tikzpicture}
\end{center}



По дереву рекурсивных вызовов получаем для $N_s$~---~
числа рекурсивных вызовов процедуры $S()$
\[
	\sum_{k=1}^{\frac{1}{3}(10^{12}-100)}{2^k} \leq N_s(10^{12}) \leq \sum_{k=1}^{10^{12}-100}{2^k} 
\]
\[
	2^{\frac{1}{3}(10^{12}-100))+1}-2 \leq N_s(10^{12}) \leq 2^{10^{12}-99}-2
\]

\section*{Задача 9}
При больших $n$ целая часть от аргумента не вносит вклад в асимптотику, поэтому рассмотрим:
\[
T(n) = nT\left(\frac{n}{2}\right) + O(n)
\]
Найдем нижнюю оценку нашей рекуренты, оценив рекуренту $G(n)$, равную C при маленьком аргументе:
\[
	G(n) = nG\left(\dfrac{n}{2}\right) =C\prod_{k=0}^{\log_{2}{n}}{\dfrac{n}{2^k}} = C\cdot \dfrac{n^{\log_{2}{n}+1}}{2^{0.5\log_{2}{n}(\log_{2}{n}+1)}} = C \cdot \dfrac{n^{\log_{2}{n}+1}}{n^{0.5(\log_{2}{n}+1})} = Cn^{0.5(\log_{2}{n}+1)}\]
	Т.е. $G(n)=  \Theta(n^{0.5(\log_{2}{n}+1)})
$~и $G(n) \leq T(n)~~\forall n \in N$

Теперь верхнюю оценку в предположении, что $n = 2^p$:
\[
	T(n) = nT\left(\dfrac{n}{2}\right)  + O(n)=n\left(T\left(\dfrac{n}{2}\right)  + O(1)\right) = n\left(\dfrac{n}{2}\left(T\left(\dfrac{n}{2^2}\right)  + O(1)\right)  + O(1)\right)=
\]
\[
	= O(1)n+ O(1)n\dfrac{n}{2}+O(1)n\dfrac{n}{2}\dfrac{n}{2^2}+\cdots + O(1)n\dfrac{n}{2}\dfrac{n}{2^2}\cdots\dfrac{n}{2^{\log_{2}{n}}} = \]
\[=O(2^p) + O(2^p2^{p-1})+ O(2^p2^{p-1}2^{p-2})+\cdots+O(2^p2^{p-1}2^{p-2}\cdots2\cdot1)=
\]
\[
	= O\left( 2^{0.5p(p+1)}\cdot \left( 2 + \sum_{k=1}^{p}{\dfrac{1}{2^{0.5k(k+1)}}} \right) \right) < O\left( 2^{0.5p(p+1)}\cdot \left( 2 + \sum_{k=0}^{p}{\dfrac{1}{2^{k}}} \right) \right)<
\]
\[
	< O\left( 2^{0.5p(p+1)}\cdot \left( 2 + \sum_{k=1}^{\infty}{\dfrac{1}{2^{k}}} \right) \right) = O\left( 2^{0.5p(p+1)}\cdot ( 2 + 2 )\right) = O(2^{0.5p(p+1)}) 
\]

Т.к. $p = \log_{2}{n}\text{,~то~~} T(n) = O(2^{0.5\log_{2}{n}(\log_{2}{n}+1)}) = O(n^{0.5(\log_{2}{n}+1)})$

В итоге, получаем, что $T(n)  = \Theta(n^{0.5(\log_{2}{n}+1)})$

\section*{Задача 8}

Дерево рекурсии с высотой $H$ будет несбалансированным для рекуренты (при больших $n$ пренебрегаем округлением)\\
 $T(n) = T(n - \sqrt{n}) + T(\sqrt{n}) + \Theta(n)$\\
 
 Поэтому оценим высоты самой правой и самой левой его ветвей, т.е. высоты деревьев рекурсии рекурент $(1)~~~T(n) = T(\sqrt{n}) + \Theta(n)$ и  $(2)~~~T(n) = T(n - \sqrt{n}) + \Theta(n)$\\
 В случае $(1)$ пусть высота равна $h$:
 \[
	n^{\left(\frac{1}{2}\right)^h} = const \Leftrightarrow h = \frac{1}{\log{2}}(\log{\log{n}} - const)~~\text{т.е.}~~h = O(\log{\log{n}})
 \]
 
 В случае $(2)$ из того, что 
 \[
	\dfrac{1}{2}n \leq n - \sqrt{n} \leq 2n~\text{при больших n, то~} n - \sqrt{n} = \Theta(n)
 \]
 Для $H$ получили оценки сверху и снизу.\\
\end{document}

