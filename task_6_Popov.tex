\documentclass[a4paper,12pt]{article} % тип документа

% report, book

%  Русский язык

\usepackage[T2A]{fontenc}			% кодировка
\usepackage[utf8]{inputenc}			% кодировка исходного текста
\usepackage[english,russian]{babel}	% локализация и переносы


\usepackage{xcolor}
\usepackage{hyperref}
 
 % Цвета для гиперссылок
\definecolor{linkcolor}{HTML}{799B03} % цвет ссылок
\definecolor{urlcolor}{HTML}{799B03} % цвет гиперссылок
 
\hypersetup{pdfstartview=FitH,  linkcolor=linkcolor,urlcolor=urlcolor, colorlinks=true}

% Математика
\usepackage{amsmath,amsfonts,amssymb,amsthm,mathtools} 


\usepackage{wasysym}
% мои добьавки-----------------------------------------
\usepackage{ dsfont }
\usepackage{listings}

%мое----------------------------------------------------
%Заговолок
\author{Попов Николай}
\title{Домашнее задание 1\LaTeX{}}

%--------------------
\usepackage{tikz}

\begin{document}

\section*{Задача 2}
$|x-y| \leq  2^n$. Если $p_1,\cdots,p_k : \forall i = 1,\cdots,k~\hookrightarrow (|x-y|~\vdots~p_i) \wedge (n~\leq~p_i~\leq~2n) \Rightarrow |x-y|\geq p_1\cdots p_k\geq n^k \Rightarrow 2^n\geq n^k \Rightarrow k \leq \ln{2}\frac{n}{\ln{n}}$
Учитывая, что при $n\rightarrow \infty \hookrightarrow \pi(n)\sim\frac{n}{\ln{n}}$, получаем оценку на вероятность ошибки \[
P_{err}\leq\dfrac{k}{\pi(2n)-\pi(n)}
\leq\dfrac{\ln{2}}{\dfrac{2\ln{n}}{\ln{2n}}-1} = \dfrac{\ln{2}\cdot\ln{2n}}{\ln{n}-\ln{2}}=\ln{2}\dfrac{\ln{2n}}{\ln{n/2}}\]\
\[
P_{err}\leq \dfrac{3}{4} \Rightarrow n \geq 2^{\frac{3+\ln{16}}{3-\ln{16}}}\approx 2^{25.384}\text{~бит}
\]
Такая длины файлов достаточна для справедливости оценки вероятности ошибки.
32 мб равны $2^82^65^6 \text{~бит}> 2^{26}$ бит $> n_0$, тогда оценка справедлива при такой длине файлов.\\
\section*{Задача 3}
$(i)~$Пусть $BPP'$ — класс языков, распознаваемых вероятностной машиной Тьюринга с вероятностью ошибки не больше чем $p<\frac{1}{2}$, работающей полиномиальное в среднем число шагов.\\
Из определения BPP следует, что $BPP \subset BPP'$, т.~к. $p = \frac{1	}{3} <\frac{1}{2} $\\

Основываясь на $\href{https://en.wikipedia.org/wiki/Hoeffding%27s_inequality}{inequality~for~Bernoulli ~random~ variables~}$покажем $BPP' \subset BPP$: пусть МТ $M' \in BPP'$, которая ошибается с вероятностью $P_{er}\leq p: p< \frac{1}{2}$. Построим с ее помощью МТ $M$, вероятность ошибки которой меньше $\frac{1}{3}$. Для этого запустим машину $M'~n$ раз подряд, сохраняя ответы (да или нет на вопрос принадлежности слова языку) и затем вернем в качетсве результата работы машины $M$ произвольный ответ из созраненных $n$ штук. Вероятность ошибки машины $M$ равна числу ошибочных ответов среди $n$  сохраненных ответов, каждый из которых, в свою очередь, является ошибочным с вероятностью $p$. Тогда вероятность ошибки машины $M = P_{er}(M) \leq \frac{1}{3} \Leftrightarrow $ тому, что число ошибочных ответов  меньше либо равно $\frac{n}{3}~(*)$. В формуле 
\[
P((\text{number of errors among n results})\leq (p-e)n)\leq exp(-2e^2n) \text{~положим~}e=\dfrac{1}{6}
\]   	

Т.~к. $p<\dfrac{1}{2}$, то $(p-e)<(0.5-e)=\dfrac{1}{3}$. Поскольку в этой формуле самое первое выражение есть вероятность ошибки, как сказано выше $(*)$, то при $exp(-2e^2n) < \dfrac{1}{3}$ получаем $BPP' \subset BPP$. Для этого достаточно запустить МТ количество раз $n > 18\ln{3}$. Например, сто раз. Т.~к.  на каждом из $n$ шагов МТ $M'$ работала в среднем полиномиально, то это свойство сохранится и для МТ $M$.\\
В итоге, $BPP = BPP'$.\\

$(ii)~$Покажем как осуществить замену полиномиальное в среднем числа шагов работы МТ на строго полиномиальное число шагов с сохранением того, что вероятность ошибки МТ строго меньше $0,5$. Пусть есть МТ, ошибающаяся с вероятностью $e < \frac{1}{2}$, которая работает полиномиальное время с вероятностью $p$ и неполиномиальное с вероятностью $1-p$. Построим по ней МТ' для того же языка, которая работает как МТ, если число шагов меньше либо равно $poly(|w|)$ ($w$ есть слово языка), а иначе прекращает моделирование МТ и возвращает произвольно ответ про принадлежность слова языку ("да" или "нет"). Вероятность ошибки МТ' равна $ep+0.5(1-p) = 0.5+p(e-0.5) < 0.5$. Т.~е. новая МТ' разрешает тот же язык за полиномиальное время и принадлежит $BPP$.\\

\section*{Задача 4}
$(i)~ABx = Cx \Leftrightarrow (AB - C)x = \overset{-}0 \Leftrightarrow Mx = \overset{-}0$, где $M = AB - C$. Умножив матрицу $M$ на произвольно выбранный вектор $x$ получим $n$ полиномов степени не больше, чем один. Если $rk (M) = n$, то вероятность того, что все полиномы зануляются выбранным вектором по лемме Шварца-Зиппеля равна $ \left(\frac{1}{N}\right)^n$, т.~к. координаты вектора могут принимать $N$ разных значений и степень многочленов равна $1$. Если ранг матрицы системы меньше $n$, то вероятность обнуления всех многочленов (а это равносильно равенству исходных матриц) будет меньше. Тогда из условия \[ \left(\frac{1}{N}\right)^n < p \Rightarrow N > \dfrac{1}{\sqrt[n]{p}}\]\\
$(ii)~x^TABx = x^TCx$. Перемножив матрицу на произвольно выьранный вектор, после переноса через знак равенства получим полином от n переменных степени не больше второй. Тогда по лемме Шварца-Зиппеля вероятность того, что полином обнуляется выбранным вектором $x$  не более $\dfrac{2}{N}$. Тогда  $\frac{2}{N} < p \Rightarrow N >\frac{2}{p}$. \\
$y^TABx = y^TCx$~---~Аналогично получаем полином степени не выше второй, но содержащий $2n$ переменных. Тогда вероятность ошибки равна $\frac{2}{2N} = \frac{1}{N}.~~~ \frac{1}{N} < p \Rightarrow N > \frac{1}{p}$.




\section*{Задача 5}
$(i)~$Пусть в графе минимальный разрез содержит $r$ ребер. Если у какой-то вершины графа степень меньше $r$, то, выделив ее и остальные вершины в два дизьюнктных подмножества множества вершин графа, получим разрез с меньшим числом ребер. Противоречие с минимальностью исходного разреза. Тогда степени всех вершин в графе не меньше $r$. Получаем \[
|E| = \dfrac{1}{2}\sum_{v \in V}{deg(v)} \geq \dfrac{1}{2}|V|k
\]
\[
P(\text{выбранное ребро в min разрезе}) = \dfrac{k}{|E|} \leq \dfrac{k}{0.5|V|k} = \dfrac{2}{|V|}
\]
$(ii)~$Алгоритм выдаст MINCUT, если в процессе его работы никакое ребро из минимального разреза не будет стянуто. Это просходит с вероятностью
\[
\left(1 - \dfrac{2}{n}\right)\left(1 - \dfrac{2}{n-1}\right)\left(1 - \dfrac{2}{n-2}\right)\cdots \left(1 - \dfrac{2}{4}\right)\left(1 - \dfrac{2}{3}\right) = \]
\[=\dfrac{n-2}{n}\cdot\dfrac{n-3}{n-1}\cdot\dfrac{n-4}{n-2}\cdots\dfrac{2}{4}\cdot\dfrac{1}{3}=\dfrac{2(n-2)!}{n!}=\dfrac{2}{n(n-1)}
\]

$(iii)~$Вероятность вероятность хотя бы одного ошибочного стягивания при каждом из $n^2$ независимых выполнений равна $1-\dfrac{2}{n(n-1)} < 1 - \dfrac{2}{n^2}$. Тогда вероятность правильно найти минимальный разрез больше 
\[
1 - \left(1-\dfrac{2}{n^2}\right)^{n^2} > 0.85 \text{~~т.~к.~~} 1 - \left(1-\dfrac{2}{n^2}\right)^{n^2} \underset{ n\rightarrow \infty}{\rightarrow}1-\dfrac{1}{e^2}\approx 0.865
\] 

\section*{Задача 6}
$(i)~$Решим задачу через графы. Для этого преобразуем каждый дизьюнкт $a\vee b~\equiv~!a\rightarrow b~\equiv~!b\rightarrow a~(*)$.

Построим по 2-КНФ граф : каждая переменная и ее отрицание будут вершинами, а ребра будут соответствовать импликациям (*).\\
Теперь заметим, что если для какой-то переменной $a$ выполняется, что из нее достижимо $!a$, а из $!a$ достижимо $a$, то задача решения не имеет, т.~к. какое бы значение для переменной $a$ мы бы ни выбрали, всегда получим противоречие ($!a=1\wedge a=1$). Тогда, для того, чтобы 2-КНФ имела решение, необходимо и достаточно, чтобы для любой переменной $a$ вершины $a$ и $!a$ находились в разных компонентах сильной связности построенного графа. Это проверим за $O(|V|+|E|)$ с помощью алгоритма поиска сильно связных компонент и последующего поиска пути между вершинами $a$ и $!a$ ($a$ произвольно).
$\href{http://e-maxx.ru/algo/2_sat}{found~at~emax}$\\
$(ii)~$Поскольку алгоритм лежит в $P$ и $P \subseteq ZPP \subseteq RP \subseteq BPP$, то он лежит во всех этих вероятностных классах.\\



\section*{Задача 7}
$(i)~$База индукции, когда под картой $a_{n-1}$ всего одна карта. Пусть под картой $a_{n-1}$ находятся $j$ карт и каждая их перестановка равновероятна. Тогда вероятность конкретной перестановки под $a_{n-1}$ картой равна $\frac{1}{j!}$. Для вставки новой карты под $a_{n-1}$ карту есть $j+1$ вариант, значит, вероятнсть получить конкретную перестановку равна $\frac{1}{j!}\cdot\frac{1}{j+1} = \frac{1}{(j+1)!}$, т.~е. снова все перестановки равновероятны.\\
$(ii)~$Аналогично пункту $(i)$.\\
$(iii)~$Пусть $T_j$~---~число итераций цикла, когда под картой $a_{n-1}$ ровно $j$ карт. Тогда общее число итераций равно $T=T_1+T_2+ \cdots +T_{n-2}$ В силу линейности матожидания : $ E_T = \sum_{j=1}^{n-2}{E_{T_j}}$
Если под картой $a_{n-1}$ находятся $j$ карт, получим    
 \[
 E_{T_j} = 1\cdot \dfrac{j+1}{n} + \cdots + k\cdot \dfrac{j+1}{n}\left(\dfrac{n-j-1}{n}\right)^{k-1} = \sum_{k=0}^{\infty}{(k+1)\dfrac{j+1}{n}\left(1-\dfrac{j+1}{n}\right)^{k}}= 
 \]
 \[
= \dfrac{j+1}{n}\left(\sum_{k=0}^{\infty}{\left(1-\dfrac{j+1}{n}\right)^{k+1}}\right)' = \dfrac{j+1}{n}\left(\dfrac{1}{1-\left(1-\dfrac{j+1}{n}\right)}\right)' = \dfrac{j+1}{n}\dfrac{1}{\left(\dfrac{j+1}{n}\right)^2}=\dfrac{n}{j+1}
 \]
 Тогда 
 \[
 E_T = \sum_{j=1}^{n-2}{\dfrac{n}{j+1}} = n~\Theta(\ln{n})\]
 Т.~к. $\ln{n}-\ln 2 = \int_{0}^{n-2}\dfrac{dx}{x+2} \leq\sum_{j=1}^{n-2}{\dfrac{1}{j+1}}\leq  \int_{0}^{n-2}\dfrac{dx}{x+1} = \ln{(n-1)}  $  \\
 
 

 
\end{document}