 \documentclass[a4paper,12pt]{article} % тип документа

% report, book

%  Русский язык

\usepackage[T2A]{fontenc}			% кодировка
\usepackage[utf8]{inputenc}			% кодировка исходного текста
\usepackage[english,russian]{babel}	% локализация и переносы


\usepackage{xcolor}
\usepackage{hyperref}
 
 % Цвета для гиперссылок
\definecolor{linkcolor}{HTML}{799B03} % цвет ссылок
\definecolor{urlcolor}{HTML}{799B03} % цвет гиперссылок
 
\hypersetup{pdfstartview=FitH,  linkcolor=linkcolor,urlcolor=urlcolor, colorlinks=true}

% Математика
\usepackage{amsmath,amsfonts,amssymb,amsthm,mathtools} 


\usepackage{wasysym}
% мои добьавки-----------------------------------------
\usepackage{ dsfont }
\usepackage{listings}

%мое----------------------------------------------------
%Заговолок
\author{Попов Николай}
\title{Домашнее задание 1\LaTeX{}}

%--------------------
\usepackage{tikz}












\begin{document}

\section*{Задача 1}
Наша цель ~---~ учеличить длину каждого дизьюнкта до 3-х, так чтобы новая КНФ была тождественно равна старой. Это выполнится, если удлинить каждый дизьюнкт так, что <<длинный>> дизьюнкт тождественно равен исходному. Добавим новую, отличную от всех имеющихся, переменную $ a$ в дизьюнкт длины $2$:\\
$(x_1 \vee x_2) \rightarrow (x_1 \vee x_2 \vee a)$\\
Если это $ a$ равно $0$, то новый дизьюнкт тождественен старому, иначе он равен $1$. Чтобы не зависеть от значения $a $, заменим:\\
$(1)~~~(x_1 \vee x_2) \equiv (x_1 \vee x_2 \vee a)\wedge(x_1 \vee x_2 \vee \overset{-}a)$\\
Если дизьюнкт длины $ 1$, повторим данную процедуру до получения тройных дизьюнктов:\\
$(2)~~~ x_1 \equiv (x_1 \vee a) \wedge (x_1 \vee \overset{-}a) \equiv (x_1 \vee a \vee b) \wedge (x_1 \vee \overset{-}a \vee b) \wedge (x_1 \vee a \vee \overset{-}b) \wedge (x_1 \vee \overset{-}a \vee \overset{-}b)$\\
\[
3\text{-}SAT \leq_p 3\text{-}CNF~~ \overset{def}\Leftrightarrow ~~\exists~ f_p(x)\equiv x: \forall x \hookrightarrow x \in 3\text{-}SAT \Leftrightarrow f_p(x) \in 3\text{-}CNF
\]
В итоге, функция $f_p$, прочитывая формулу из входа, будет создавать новую формулу, которая тождественна исходной и длиннее её максимум в $23$ раза (посимвольно), заменяя каждый дизьюнкт по правилам (1) и (2), т.~е. $f_p$ является полиномиальной. Ввиду того, что выполнимость сгенерированной формулы равносильна выполнимости исходной формулы, то необходимая сводимость показана.\\

\section*{Задача 2}
$(i)$~$\psi = x_1 \vee x_2 \vee \overset{-}x_3$~~~$A_{\psi} = \{\{x_1,\overset{-}x_1\},\{\{x_2,\overset{-}x_2\},\{\{x_3,\overset{-}x_3\},\{x_1,x_2,\overset{-}x_3\}\}$\\
Протыкающее множество~---~$\{x_1,x_2,\overset{-}x_3\}$\\
$(ii)$~$A_{\chi} = \{ \{x_1,\overset{-}x_1\},\{x_2,\overset{-}x_2\},\{x_1,x_2\},\{x_1,\overset{-}x_2\},\{\overset{-}x_1\}\}$ \\

Дополняем КНФ до тождественной добавив дизьюнкты вида $a\vee \overset{-}a$ для каждой из n переменных.\\ 

Пусть КНФ выполнима. 
Будем выбирать из каждого дизьюнкта в преобразованной КНФ по одной переменной ($x$ или $\overset{-}x$), которая равна единице в выполняющем наборе и помещать их в наше протыкающее множество. Т.~к. для каждой переменной  есть дизьюнкт $a\vee \overset{-}a$, то из каждой такой пары либо переменная, либо ее отрицание попадет в протыкающее множество, т.е. в нем будет n элементов, т.~к. из выполнимости КНФ следует её непротиворечивость, т.е. нет пар $a, \overset{-}a$. \\

Пусть у нас есть протыкающее множество из n элементов. Если в нем есть пара $a, \overset{-}a$, то для какой то из $n$ переменных в этом протыкающем множестве нет ни ее, ни ее отрицания, что невозможно, т.~к. тогда протыкающее множество не пересекается с дизьюнктом где есть эта переменная в паре со своим отрицанием.  Т.~е. пар $a, \overset{-}a$ нет в нашем протыкающем множестве, значит, в нем для каждой переменной есть либо она, либо ее отрицание и это множество пеерсекается с каждым дизьюнктом, обозначим (1). Пусть каждая переменная их протыкающего множества равна $1$. В силу (1) получаем, что протыкающее множество - это выполняющий набор для КНФ, т.к. каждый диьюнкт будет равен $1$ и противоречия в таком наборе нет, т.е. КНФ выполнима.\\


\section*{Задача 3}
$(i)$~Граф строится однозначно, поэтому не буду повторяться. $n_{new}(\psi)+2m_{new}(\psi) = 5$. Вершинное покрытие $\{x_1,  \overset{-}x_2, \overset{-}x_3, x_2, x_3\}$\\

\noindent 1)Дополненная по правилам из задачи №1 КНФ тождественна полученной ровно-3-КНФ.\\
 
\noindent 2)Пусть ровно 3 КНФ выполнима. Тогда сущесвует выполняющий непротиворечивый набор, при котором в каждом дизьюнкте, а значит, и треугольнике, ему соответствующем, в графе ровно по одной веришине равной $1$. Добавим вершину графа из треугольников в вершинное покрытие, если значение вершины $0$, и из пар $a\text{---}\overset{-}a$, если значение вершины равно $1$. Т.к. набор непротиворечив, то среди пар будут отмечены ровно $n$ вершин. В каждом треугольнике отмечено по две вершины в каждом из $m$ треугольников. Каждое ребро при таком выборе вершин инцидентно хотя бы одной  
выбранной вершине, т.~е. выбранное множество вершин~---~это вершинное покрытие и в нем есть $n+2m$ веришины.\\

$(*)$~Пусть дано вершинное покрытие мощности $n+2m$. Оно таково, что из каждого дизьюнктого треугольника в него входят хотя бы две вершины, т.~к. иначе одно из ребер, образующих треугольник, не будет иметь ни одной инцидентной вершины, входящей в вершинное покрытие. Пусть в каждом треугольнике ровно по две вершины входят в покрытие. Посчитаем их равными $0$. Из пары $a\text{---}\overset{-}a$, можем выбрать еще ровно по $1$ вершине, которые посчитаем равными $1$. Если из какой-то пары включим в покрытие обе вершины, то останется пара вершин $a\text{---}\overset{-}a$, ребро между которыми не будет иметь ни одной инцидентной вершины из вершинного покрытия, что невозможно.\\ 
 
Если хотя бы в одном треугольнике все три вершины входят в вершинное покрытие, то останется не больше $n - 1$ места для $2n$ парных вершин $a\text{---}\overset{-}a$ графа. Тогда останется какая-то пара вершин $a\text{---}\overset{-}a$, ребро между которыми не будет иметь ни одной инцидентной вершины из вершинного покрытия, что невозможно.\\
Получаем, что набор переменных реализуемый данным вершинным покрытием будет выполняющим набором для соответствующей КНФ~$(*)$.\\

\noindent$(ii)$~Для графа $G_{\chi}~~n_{new}(\psi)+2m_{new}(\psi) = 20$. Пусть есть вершинное покрытие мощности 20. Из $(*)\cdots(*)$ получаем противоречие, как и для меньшего числа вершин, ч.~т.~д.  

\section*{Задача 4}
$(i)~m = 1$, в графе $\tilde{G_{\psi}}$ три одиночные вершины и есть клика размера $1$.\\
$(ii)$~ Предположим, в этом графе есть клика размера  $8$. Т.к. ребра соединяют вершины из разных долей (т.е. разных дизьюнктов), которые не  являются отрицанием друг друга, то приняв за 1 значения этих верщин получим непротиворечивый выполняющий набор для КНФ $\chi$. Но она невыполнима, значит, нет указанной клики и клики большего размера, т.к. из существования таковой следует существование клики размера 8.\\

Доказательство сводимости полностью аналогично подобному с семинара ($ CNF \leq CLIQUE$), дизьюнктам соответствуют доли в графе, число долей равно числу вершин в клике.\\

\section*{Задача 5}

$(i)$~$\psi = x_1 \vee x_2 \vee \overset{-}x_3$\\
$\tilde{\psi} = x_1\wedge x_2\wedge \overset{-}x_3\wedge(x_1\vee\overset{-}x_2)\wedge(\overset{-}x_1\vee x_3)\wedge(\overset{-}x_2\vee x_3)\wedge(\overset{-}x_1\vee\overset{-}d)\wedge(x_2\vee\overset{-}d)\wedge(\overset{-}x_3\vee \overset{-}d)$\\
\\$\begin{matrix}
x_1 & x_2 & \overset{-}x_3 & max \text{~кол-во выполнимых дизьюнктов}\\
0 &0 &0 & 6 ~(d=0)\\ 
0 &0 &1& 7 ~(d=0)\\
0& 1 &0& 7 ~(d=0)\\
0 &1 &1 & 7 ~(d=0)\\
1 &0 &0 & 7 ~(d=0)\\
1 &0 &1& 7~ (d=0)\\
1 &1 &0& 7~ (d=0)\\
1 &1& 1& 7 ~(d=0)\\

\end{matrix}
$\\
$(ii)$~По таблице получаем, что получить больше $kq$ выполнимых дизьюнктов нельзя.\\

 Инвариант преобразования~---~максимальное число выполнимых дизьюнктов среди десяти полученных $q = 7$, если соответствующий исходный дизьюнкт выполним, и равно $6$ иначе.\\
 
Сводимость: каждый дизьюнкт исходной РОВНО-3-КНФ преобразуем по правилу из файла NPc в 10 дизьюнктов новой КНФ. По таблице получаем, что выполнимость исходного дизьюнкта равносильна тому, что число новых выполнимых дизьюнктов при подстановке нужного $d$ равно 7, иначе равно 6. Отсюда получаем, если 3-КНФ из $n$ дизьюнктов выполнима $\Leftrightarrow$ выполнимы все дизьюнкты в ней и всего выполнимых дизьюнктов в 2-КНФ будет $7n$, иначе если 
исходная КНФ невыполнима $\Leftrightarrow$ в ней есть хотя бы один невыполнимый дизьюнкт $\Leftrightarrow$ всего выполнимых дизьюнктов в 2-КНФ будет не больше $7n - 1$. Т.~к. приведение к РОВНО-3-КНФ из 3-КНФ полиномиально от количества символов в исходной формуле и полиномиальны построение 2-КНФ и подсчет выполнимых дизьюнктов в ней, то получена нужная полиномиальная сводимость.\\

\section*{Задача 7}
	 
Найдем в каком случае построенная в приведении $Circuit\text{-}SAT \leq 3\text{-}CNF$ формула выполнима. Когда она равна $1$, то $y_n = 1$. При этом дизьюнкт $(y_n = \cdots)$ равен $1$, когда равна единице правая часть и т.~д. Получается, пытаясь найти выполняющий набор для этой формулы, мы переходя от дизьюнкта к дизьюнкту, разворачиваем формулу, которая равносильна схеме и найдем её выполняющий набор, являющийся частью искомого. Обратно, если есть набор выполяющий формулу, равносильную схеме, то чтобы получить выполняющий набор для построенной формулы, всем остальным переменным нужно присвоить $1$. Получили, что эти две формулы равновыполнимы.\\

\section*{Задача 6}
Основано на идее из ответа отсюда $\href{https://stackoverflow.com/questions/26851829/3-colouring-of-a-graph-polynomial-time}{mystackoverflow}$\\
Если граф можно раскрасить, добавим три новые вершины с тремя разными отметками. Переберем для каждой вершины три возможных цвета, подсоединив к ней две какие-то вершины из нового треугольника и запустим процедуру проверки. Найдя подходящий цвет преходим к следующей вершине. Т.~к. раскраска исходного графа есть, то добавление треугольника и двух ребер не изменит ее. Полиномиальных проверок будет полиномиальное число от числа вершин исходного графа. Т.~е. получен нужный нам алгоритм.\\  
 
\end{document}